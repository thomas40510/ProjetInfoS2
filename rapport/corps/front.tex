\section*{Introduction}
  Pour mener une étude océanique, il est important de commencer par caractériser les masses d'eau.
  On étudie, à un endroit ou un volume donné, la composition et les caractéristiques de ce volume d'eau.
  Celles-ci peuvent être physiques (température, salinité), chimique (pH, espèces présentes, minéraux) ou biologiques (espèces vivantes présentes).
  Toutes ces caractéristiques sont appelées les \emph{traceurs océaniques}.
  De plus, on utilise, pour modéliser la circulation océanique, des modèles numériques offrant -- grâce aux progrès des méthodes et outils de calcul -- de nouvelles modélisations de plus en plus performantes des océans.

  Ainsi, les traceurs océaniques sont \textit{de facto} un moyen de vérifier la performance de ces nouvelles solutions numériques~\cite{jean-baptiste_circulation_1994}.

  %Ce rapport propose donc, dans un premier temps, un tour d'horizon des différentes catégories de traceurs qu'il est possible d'exploiter, avant de porter une étude plus détaillée sur deux catégories spécifiques et intrinsèquement liées.
  Ce rapport propose donc un tour d'horizon des deux grandes catégories de traceurs utilisés dans le suivi des océans.

\section*{Mots-clés}
  Traceurs, océan, pH, plancton, isotopes, carbone.

\selectlanguage{english}
\section*{Abstract}
  Oceanic tracers are the best insights of a part of the ocean and are therefore used as indicators by researchers.
  Given the constant improvements in computational methods applied to the oceanic circulation, becoming more and more efficient and realistic, tracers are essential to verify the accuracy of such models with real data.

  There are numerous types of tracers, while the most commonly used are physical -- such as temperature or salinity -- chemical (pH and concentrations of ions gases and metals) and biological, \ie{} the list of any living species found at a given place.

  The following report will then address the two most common categories of oceanic tracers.
  %establish a list of the different categories of tracers, before addressing two of them more precisely.

\section*{Keywords}
  Tracers, ocean, pH, plankton, isotopes, carbon.

\selectlanguage{french}

  \vspace{3em}
\begin{center}
  \begin{Large}
* \\
*  \ \ *
\end{Large}
\end{center}



